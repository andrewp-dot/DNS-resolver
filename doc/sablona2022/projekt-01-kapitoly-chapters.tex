% Tento soubor nahraďte vlastním souborem s obsahem práce.
%=========================================================================
% Autoři: Michal Bidlo, Bohuslav Křena, Jaroslav Dytrych, Petr Veigend a Adam Herout 2019

% Pro kompilaci po částech (viz projekt.tex), nutno odkomentovat a upravit
%\documentclass[../projekt.tex]{subfiles}
%\begin{document}

%===============================================================================

\newcommand*{\addNewChapter}[2][1]{\addtocounter{chapterCounter}{#1}\chapter{#2}\label{#2}}


\addNewChapter{Úvod}
Komunikácia je neodmysliteľnou súčasťou pre zdielanie a prenos informácii. V dnešnej dobe sa na tento účel využíva hlavne výpočtová technológia.

Základ počítačovej komunikácie tvoria dve hlavné zložky: adresovanie a smerovanie. Adresovanie je spôsob vytvárania a priraďovania adries počítačom. 
Adresa je jednoznačný údaj, ktorý presne identifikuje práve jeden adresovateľný prvok. Smerovanie je proces výberu cesty pre prevádzku v sieti medzi viacerými sieťami. \cite{Smerovanie}

Adresa zariadení na úrovni TCP/IP je vo formáte IP adresy (napríklad 95.82.140.207). 
Pre človeka je adresovanie na úrovni počítačov komplikované. Adresovanie počítačov pomocou IP adries nie je pre ľudí dobré riešenie, pretože sú náročné na zapamätanie.
Z toho dôvodu bol vyvinutý systém, ktorý umožňuje adresovať zariadenia pomocou doménových adries.

\section{Domain name system}
\label{Domain name system}
DNS (angl. domain name system) je systém, ktorého cieľom je poskytnúť mechanizmus na pomenovávanie
zdrojov (zariadení) takým spôsobom, aby tieto mená boli použiteľné v rôznych zariadeniach, sieťach, protokolových rodinách, 
na internete a administratívnych organizáciach. \cite{rfc1035} 

\section{Architektúra DNS}
Základnou úlohou služby DNS je mapovanie doménových adries (tzv. doménových mien) na IP adresy. 
Systém DNS sa skladá z troch hlavných častí - priestoru doménových mien, DNS serverov a resolveru. Priestor doménových mien je databáza, kde sú dané doménové mená uložené. 
Táto databáza je hierarchicky usporiadaná do stromovej štruktúry (pre rýchle vyhladávanie konkrétnych domén). \cite{Matoušek}
Z tohoto dôvodu majú doménové mená hierarchiu. Prekladajú sa od koreňa a tzv. top level domains (skratka TLD - napríklad en., sk., cz., com.)



\section{DNS resolver}
\label{DNS resolver}






\section{Formát správy}
\label{Formát správy}

% \subsection{QCLASS}

\section{Dotazovanie}
\label{Dotazovanie}


\section{Formát odpovede}

\section{IPv6}


% Pro kompilaci po částech (viz projekt.tex) nutno odkomentovat
%\end{document}

\addNewChapter{Návrh programu}
Pre realizáciu programu bol použitý jazyk C++. Pre preklad bol použitý nástroj GNU Make 3.81. Program je vyvíjaný podľa štandardu c++17.

/*
 * stub resolver
 * objektový nárvh / diagram
 * spustenie
 * výstup   - standard
            - chybové stavy
 */



\addNewChapter{Popis programu}
/*
 * kompatibilita
 * popis Častí kódu 
 * odosielanie, prijatie 
 */


\addNewChapter{Testovanie}
/*
 * testovacie prostredie
 * spôsoby testovania
 * použité nástroje - python3
                    - dig 
 */


