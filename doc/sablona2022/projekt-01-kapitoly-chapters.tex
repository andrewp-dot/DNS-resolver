% Tento soubor nahraďte vlastním souborem s obsahem práce.
%=========================================================================
% Autoři: Michal Bidlo, Bohuslav Křena, Jaroslav Dytrych, Petr Veigend a Adam Herout 2019

% Pro kompilaci po částech (viz projekt.tex), nutno odkomentovat a upravit
%\documentclass[../projekt.tex]{subfiles}
%\begin{document}

%===============================================================================

\newcommand*{\addNewChapter}[2][1]{\addtocounter{chapterCounter}{#1}\chapter{#2}\label{#2}}


\addNewChapter{Úvod}
Komunikácia je neodmysliteľnou súčasťou pre zdielanie a prenos informácii. V dnešnej dobe sa na tento účel využíva hlavne výpočtová technológia.

Základ počítačovej komunikácie tvoria dve hlavné zložky: adresovanie a smerovanie. Adresovanie je spôsob vytvárania a priraďovania adries počítačom. 
Adresa je jednoznačný údaj, ktorý presne identifikuje práve jeden adresovateľný prvok. Smerovanie je proces výberu cesty pre prevádzku v sieti medzi viacerými sieťami. \cite{Smerovanie}

Adresa zariadení na úrovni TCP/IP je vo formáte IP adresy (napríklad 95.82.140.207). 
Pre človeka je adresovanie na úrovni počítačov komplikované. Adresovanie počítačov pomocou IP adries nie je pre ľudí dobré riešenie, pretože sú náročné na zapamätanie.
Z toho dôvodu bol vyvinutý systém, ktorý umožňuje adresovať zariadenia pomocou doménových adries.

\section{Domain name system}
\label{Domain name system}
DNS (angl. domain name system) je systém, ktorého cieľom je poskytnúť mechanizmus na pomenovávanie
zdrojov (zariadení) takým spôsobom, aby tieto mená boli použiteľné v rôznych zariadeniach, sieťach, protokolových rodinách, 
na internete a administratívnych organizáciach. \cite{rfc1035} 

\section{Architektúra DNS}
Základnou úlohou služby DNS je mapovanie doménových adries (tzv. doménových mien) na IP adresy. 
Systém DNS sa skladá z troch hlavných častí - priestoru doménových mien, DNS serverov a resolveru. Priestor doménových mien je databáza, kde sú dané doménové mená uložené. 
Táto databáza je hierarchicky usporiadaná do stromovej štruktúry (pre rýchle vyhladávanie konkrétnych domén). \cite{Matoušek}

\section{DNS resolver}
\label{DNS resolver}
Je to klientský program, ktorý získavá informácie zo systému DNS prostredníctvom dotazovania sa na DNS servery. Tento proces sa nazýva DNS rezolúcia.
Resolver môže mať viac typov konfigurácii. 
V tejto práci sa budeme zaoberať nárvhom a implementáciou tzv. stub resolvera.


\addNewChapter{Návrh programu}

\section{Stub resolver}
\label{Stub resolver}
Stub resolver je typ resolvera, ktorý interaguje s aplikáciou alebo užívateľom a rekurzívnym DNS serverom. Samotný resolver rezolúciu nevykonáva.
V tomto prípade sa rezolúcia realizuje zaslaním dotazu resolvera na rekurzívny DNS server, ktorý odošle odpoveď na dotaz.
Prijatú odpoveď resolver spracuje a získané informácie poskytne aplikácii alebo zobrazí užívateľovi.


\section{Dotazovanie}
\label{Dotazovanie}

Dotazovanie na server prebieha zasielaním správ. Užívateľ alebo aplikácia poskytne údaje resolveru (ako sú dotazovaná adresa, typ dotazu).
Resolver dané údaje vloží do správy, ktorú odošle na DNS server.

\subsection{Formát správy}
\label{Formát správy}

% \subsection{QCLASS}



\subsection{Formát odpovede}


\section{Použité technológie}
Pre realizáciu programu bol použitý jazyk C++. Pre preklad bol použitý nástroj GNU Make 3.81. Program je vyvíjaný podľa štandardu c++17.


% Pro kompilaci po částech (viz projekt.tex) nutno odkomentovat
%\end{document}



/*
 * stub resolver
 * objektový nárvh / diagram
 * spustenie
 * výstup   - standard
            - chybové stavy
 */



\addNewChapter{Popis programu}
/*
 * kompatibilita
 * popis Častí kódu 
 * odosielanie, prijatie 
 */


\addNewChapter{Testovanie}
/*
 * testovacie prostredie
 * spôsoby testovania
 * použité nástroje - python3
                    - dig 
 */


