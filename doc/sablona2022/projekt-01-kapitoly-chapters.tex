% Tento soubor nahraďte vlastním souborem s obsahem práce.
%=========================================================================
% Autoři: Michal Bidlo, Bohuslav Křena, Jaroslav Dytrych, Petr Veigend a Adam Herout 2019

% Pro kompilaci po částech (viz projekt.tex), nutno odkomentovat a upravit
%\documentclass[../projekt.tex]{subfiles}
%\begin{document}

%===============================================================================

\newcommand*{\addNewChapter}[2][1]{\addtocounter{chapterCounter}{#1}\chapter{#2}\label{#2}}


% \chapter{Úvod}
% \label{Úvod}
\addNewChapter{Úvod}


\section{Domain name system}
\label{Domain name system}
DNS (angl. domain name system) je systém, ktorého cieľom je poskytnúť mechanizmus na pomenovávanie
zdrojov (zariadení) takým spôsobom, aby tieto mená boli použiteľné v rôznych zariadeniach, sieťach, protokolových rodinách, 
na internete a administratívnych organizáciach. [RFC1035]

Princíp DNS:

/*
 * popis problému
 * spôsob fungovania DNS
 * DNS resolver
 */




\section{Formát správy}
\label{Formát správy}

% \subsection{QCLASS}

\section{Formát odpovede}

\section{IPv6}


% Pro kompilaci po částech (viz projekt.tex) nutno odkomentovat
%\end{document}

\addNewChapter{Návrh programu}
Pre realizáciu programu bol použitý jazyk C++. Pre preklad bol použitý nástroj GNU Make 3.81. Program je vyvíjaný podľa štandardu c++17.

/*
 * stub resolver
 * objektový nárvh / diagram
 * spustenie
 * výstup   - standard
            - chybové stavy
 */



\addNewChapter{Popis programu}
/*
 * kompatibilita
 * popis Častí kódu 
 * odosielanie, prijatie 
 */


\addNewChapter{Testovanie}
/*
 * testovacie prostredie
 * spôsoby testovania
 * použité nástroje - python3
                    - dig 
 */

\cite[Hello]{Pravidla}

