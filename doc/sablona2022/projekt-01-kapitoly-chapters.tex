% Tento soubor nahraďte vlastním souborem s obsahem práce.
%=========================================================================
% Autoři: Michal Bidlo, Bohuslav Křena, Jaroslav Dytrych, Petr Veigend a Adam Herout 2019

% Pro kompilaci po částech (viz projekt.tex), nutno odkomentovat a upravit
%\documentclass[../projekt.tex]{subfiles}
%\begin{document}

%===============================================================================

\newcommand*{\addNewChapter}[2][1]{\addtocounter{chapterCounter}{#1}\chapter{#2}\label{#2}}


\addNewChapter{Úvod}
Komunikácia je neodmysliteľnou súčasťou zdielania a prenosu informácii. V súčasnosti sa na tento účel využíva hlavne výpočtová technológia.

Základ počítačovej komunikácie tvoria dve hlavné zložky: adresovanie a smerovanie. Adresovanie je spôsob vytvárania a priraďovania adries počítačom. 
Adresa je jednoznačný údaj, ktorý presne identifikuje práve jeden adresovateľný prvok. Smerovanie je proces výberu cesty pre prevádzku v sieti medzi viacerými sieťami. \cite{Smerovanie}

Adresa zariadení na úrovni TCP/IP je vo formáte IP adresy (napríklad 95.82.140.207). 
Pre človeka je adresovanie na úrovni počítačov komplikované. Adresovanie počítačov pomocou IP adries nie je nutne dobré riešenie pre užívateľov, pretože sú náročné na zapamätanie.
Z toho dôvodu bol vyvinutý systém, ktorý umožňuje adresovať zariadenia pomocou doménových adries.

\section{Domain name system}
\label{Domain name system}
DNS (angl. domain name system) je systém, ktorého cieľom je poskytnúť mechanizmus na pomenovávanie
zdrojov (zariadení) takým spôsobom, aby tieto mená boli použiteľné v rôznych zariadeniach, sieťach, protokolových rodinách, 
na internete a administratívnych organizáciach. \cite{rfc1035} 

\section{Architektúra DNS}
Základnou úlohou služby DNS je mapovanie doménových adries (tzv. doménových mien) na IP adresy. 
Systém DNS sa skladá z troch hlavných častí - priestoru doménových mien, DNS serverov a resolveru. Priestor doménových mien je databáza, kde sú dané doménové mená uložené. 
Táto databáza je hierarchicky usporiadaná do stromovej štruktúry (pre rýchle vyhladávanie konkrétnych domén). \cite{Matoušek}

\section{DNS resolver}
\label{DNS resolver}
Je to klientský program, ktorý získavá informácie zo systému DNS prostredníctvom dotazovania sa na DNS servery. Tento proces sa nazýva DNS rezolúcia.
Resolver môže mať viac typov konfigurácii. 
Táto práca je zameraná na nárvh a implementáciou tzv. stub resolvera.


\addNewChapter{Návrh programu}

\section{Stub resolver}
\label{Stub resolver}
Stub resolver je typ resolvera, ktorý interaguje s aplikáciou alebo užívateľom a rekurzívnym DNS serverom. Samotný resolver rezolúciu nevykonáva.
V tomto prípade sa rezolúcia realizuje zaslaním dotazu resolvera na rekurzívny DNS server, ktorý odošle odpoveď na dotaz.
Prijatú odpoveď resolver spracuje a získané informácie poskytne aplikácii alebo zobrazí užívateľovi.

\section{Dotazovanie}
\label{Dotazovanie}

Dotazovanie na server prebieha zasielaním správ. Užívateľ alebo aplikácia poskytne údaje resolveru (ako sú dotazovaná adresa, typ dotazu).
Resolver dané údaje vloží do správy, ktorú odošle na DNS server, na čo DNS server odpovie. Komunikácia prebieha štandartne cez protokol UDP. Jeden dotaz zodpovedá jednému UDP datagramu.

\subsection{Formát správy}
\label{Formát správy}

Správa sa skladá z 5 hlavných častí: hlavičky, otázky, odpovednej sekcie, autoritatívnej sekcie a dodatočnej sekcie. 
V hlavičke sú uložené údaje o dotaze, ako ID, typ dotazu, príznaky (dodatočné informácie o správe), kód chyby a čítače záznamov pre jednotlivé sekcie.
Časť otázky sa skladá z dotazovanej adresy rozloženej na časti (tzv. labels), typu dotazu a triedy dotazu (typicky IN pre internet). 

\newpage

\subsubsection{Príznaky}
\begin{itemize}
    \item QR - tento príznak odlišuje dotaz (0) a odpoveď (1)
    \item OPCODE - typ dotazu
    \item AA - ak je nastavený na hodnotu 1, tak odpoveď je autoritatívna (dotazovaný server je autorita pre dotazovanú doménu)
    \item TC - ak je nastavený na hodnotu 1, tak odpoveď je skrátená
    \item RD - ak je nastavený na hodnotu 1, tak je požadovaná rekurzívna rezolúcia
    \item RA - určuje, či je rekurzívna rezolúcia implementovaná v odpovedi
    \item Z - príznak rezervovaný pre budúce použitie (musí byť nulový)
    \item RCODE - kód odpovede
\end{itemize}

Príznaky predstavujú základné informácie o obsahu správy.
Pri zasielaní dotazu sa nastavujú parametre v hlavičke a otázkovej sekcii. Zvyšné sekcie sú určené pre zaznamenanie odpovede. Hodnoty príznakov
QR, AA, TC, RA, Z, RCODE musia byť pri zaslaní dotazu nulové.

\newpage
\section{Objektový nárvh}
\label{Objektový návrh}

\begin{figure}[H]
    \includegraphics[width=\textwidth]{./template-fig/DNS_resolver.jpg}
    \caption{Class diagram}
    \centering
\end{figure}

Program sa skladá z 5 hlavných tried: InputParser, Query, Message, Connection a Error. Prvé 4 triedy spravujú správy a trieda Error spracováva chybové stavy, chybové hlásenia a návratovú hodnotu programu.

Trieda InputParser má za úlohu spracovať argumenty zadané používateľom a vytvoriť dotaz. Dotaz reprezentuje trieda Query.
Spracovanie dotazu, jeho zaslanie a prijatie odpovede má za úlohu trieda Connection. Trieda Message slúži na vytvorenie správy z dotazu a jej konverziu medzi 
formátom, ktorý sa odosiela po sieti a formátom, ktorý spracováva aplikácia.


\addNewChapter{Popis programu}
/*
 * chovanie programu v prípade nedostania
 * kompatibilita
 * popis Častí kódu 
 * spustenie
 * výstup   - standard
            - chybové stavy
 */


% \section{Použité technológie}
% Pre realizáciu programu bol použitý jazyk C++. Pre preklad bol použitý nástroj GNU Make 3.81. Program je vyvíjaný podľa štandardu c++17. Program je navrhnutý 
% primárne pre systémy typu UNIX.

\addNewChapter{Testovanie}
/*
 * testovacie prostredie
 * spôsoby testovania
 * použité nástroje - python3
                    - dig 
 */



